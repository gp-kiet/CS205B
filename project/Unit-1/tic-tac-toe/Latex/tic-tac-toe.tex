\documentclass{beamer}
\usepackage{amsmath}
\usepackage{graphicx}
\usepackage{listings}
\usepackage{xcolor}
\usetheme{Madrid}
\usepackage[most]{tcolorbox}

\title{Experiment-1: Tic-Tac-Toe with Minimax}
\subtitle{Based on Python Implementation and Documentation}
\author{Prof. Gaurav Parashar \\
Department of Computer Science \& Engineering \\
KIET Group of Institutions, Ghaziabad
}
\date{07/08/2025}

% Define Python style for listings
\definecolor{codegray}{rgb}{0.5,0.5,0.5}
\definecolor{backcolor}{rgb}{0.95,0.95,0.95}

\lstdefinestyle{pythonstyle}{
    language=Python,
    backgroundcolor=\color{backcolor},
    commentstyle=\color{gray}\ttfamily,
    keywordstyle=\color{blue},
    numberstyle=\tiny\color{codegray},
    stringstyle=\color{red},
    basicstyle=\ttfamily\scriptsize,  % Reduced font size
    breaklines=true,
    frame=single,
    numbers=left,
    numbersep=5pt,
    showstringspaces=false,
    tabsize=2
}

\begin{document}

% Title Slide
\begin{frame}
  \titlepage
\end{frame}

% Overview
\begin{frame}{Overview}
  \tableofcontents
\end{frame}


\section{Problem Statement}
\begin{frame}{Information}
\begin{tcolorbox}[colback=blue!5!white,colframe=blue!75!black,title=Problem Statement]
Develop an AI to play Tic-Tac-Toe using the Minimax algorithm.
\end{tcolorbox}

\begin{tcolorbox}[colback=yellow!5!white,colframe=yellow!50!black,
  colbacktitle=yellow!75!black,title=Implementation Details]
Represent game as 3x3 board, apply recursive minimax logic with pruning.
\end{tcolorbox}
\begin{tcolorbox}[colback=red!5!white,colframe=red!50!black,
  colbacktitle=red!75!black,title=Plan]
This program is divided into two phases Game Setup with human vs human and human vs AI.
\end{tcolorbox}
\end{frame}

\section{Phase-1}
\begin{frame}
\centering
\Huge \textcolor{blue}{PHASE-1}
\end{frame}



\subsection{Introduction}
\begin{frame}{Game Setup}
\begin{itemize}
  \item 3x3 Board using NumPy Array
  \item Player 1 Symbol: \texttt{X}
  \item Player 2 Symbol: \texttt{O}
  \item Goal: Place three of the same symbols in a row, column, or diagonal
\end{itemize}
\end{frame}

\subsection{Algorithm Steps}
\begin{frame}{Step 1: Initialization}
\begin{itemize}
  \item Create empty 3x3 board with '-' symbols
  \item Assign symbols: \texttt{p1s = 'X'}, \texttt{p2s = 'O'}
\end{itemize}
\end{frame}

\begin{frame}{Step 2: Game Loop}
\begin{itemize}
  \item Loop for 9 turns (maximum in 3x3 grid)
  \item Even turn: Player 1 (\texttt{X})
  \item Odd turn: Player 2 (\texttt{O})
\end{itemize}
\end{frame}

\begin{frame}{Step 3: Place Symbol}
\begin{itemize}
  \item Prompt player to input row and column (1-3)
  \item Validate input: in bounds and cell is empty
  \item Place symbol and display updated board
\end{itemize}
\end{frame}

\begin{frame}{Step 4: Check Win}
\begin{itemize}
  \item Check if player symbol appears in:
  \begin{itemize}
    \item Any row (check\_rows)
    \item Any column (check\_cols)
    \item Any diagonal (check\_diagonals)
  \end{itemize}
  \item If true, declare the winner and end game
\end{itemize}
\end{frame}

\begin{frame}{Step 5: Draw Condition}
\begin{itemize}
  \item After 9 turns, if no player has won
  \item Check both players for win one last time
  \item If none, declare the result as Draw
\end{itemize}
\end{frame}

\subsection{Functions Overview}
\begin{frame}{Function: place(symbol)}
\begin{itemize}
  \item Input row and column from user
  \item Validate and place symbol
  \item Print updated board
\end{itemize}
\end{frame}

\begin{frame}{Function: check\_rows, check\_cols, check\_diagonals}
\begin{itemize}
  \item Each checks if symbol appears three times
  \item Returns \texttt{True} if win is detected
\end{itemize}
\end{frame}

\begin{frame}{Function: won(symbol)}
\begin{itemize}
  \item Combines row, column, and diagonal checks
  \item Used after every move to check win
\end{itemize}
\end{frame}

\subsection{Conclusion}
\begin{frame}{Conclusion}
\begin{itemize}
  \item A simple turn-based 2-player game
  \item Efficient input-validation and win-checking
  \item Demonstrates control flow, loops, and conditionals
\end{itemize}
\end{frame}

\begin{frame}{Thank You}
  \centering
  Questions?
\end{frame}


\section{Phase-2}
\begin{frame}
\centering
\Huge \textcolor{blue}{PHASE-2}
\end{frame}



% =====================
% Appendix with Code
% =====================
\appendix
\section{Appendix: Source Code}
\begin{frame}[allowframebreaks]{Python Code with Documentation}
\lstinputlisting[style=pythonstyle]{game.py}
\end{frame}

\end{document}